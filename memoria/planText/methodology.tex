\chapter{\label{ch:methodology}Methodology}

This section needs to answer self-imposed questions and should reflect
that you have a good understanding of the problem and the barriers
that you may find in the path.



Some of the questions that should be answered include:
%
\begin{itemize}
\item What are the projects constraints (if any)?
\item What are the technical challenges and uncertainties?
\item What is your preferred approach and why?  
\item Explain your methodology to conduct the research and to obtain
  the stated objectives.
\end{itemize}
%

When relevant, you can have a section dedicated to the methods and
techniques and another to the facilities that you expect to use during
your studies.

\section{\label{sec:methods}Methods and techniques}

Explain how your approach to collecting and analysing information will
help you satisfy your aims and objectives. Potential data collection
methods and possible analytical techniques give a sense of the
direction of your research. Explain the choices behind case study
organisations or locations, as well as sampling strategies or
particular computer-based techniques.

\section{\label{sec:facilities}Facilities}

Explain the facilities to be used.  Some examples include:
%
\begin{itemize}
\item Is all the necessary hardware/software in place?
\item If not, how will it be acquired and how long will it take to put
  everything in place?
\item Does it have any budget implications?  Are they covered by PhD
  fundings?
\end{itemize}
%

If budget is important for your research project, then I recommend you
to dedicate a section to it within Chapter~\ref{ch:schedule} (see
Section~\ref{sec:budget}).
