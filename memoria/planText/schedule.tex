\chapter{\label{ch:schedule}Programme schedule}

You do not need to produce a detailed time plan because research
projects always tend to evolve.  However, it is extremely useful to
explain in general terms what you are proposing to do, and when, in
order to get a sense of the scale of the task.  This is especially
important if you are proposing to undertake case study work or
fieldwork.

Remember that you will be expected to submit your thesis within three
years (six years for part-time students) so it is important to show a
feasible timeline.

You can divide this chapter into two sections if necessary: e.g.,
Timeline and Budget.

\section{\label{sec:timeline}Timeline}

You can include a Gantt chart like the one shown in
Fig.~\ref{fig:Gannt-chart}.  It will be very much appreciated by your
examiners.
%
\begin{sidewaysfigure}
\begin{ganttchart}[
   vgrid={*{11}{gray, dotted}, *1{black, dashed}},
   bar label node/.append style={
     align=left,
     text width=width("Obj. 2: Hardware portability?")}
   ]{1}{36}
\gantttitle{Year 1}{12} \gantttitle{Year 2}{12} \gantttitle{Year 3}{12} \\
\ganttbar{Obj. 1: Migration}{1}{8} \\
\ganttbar{Obj. 2: Software verification}{6}{12} \\
\ganttbar{Obj. 3: Hardware portability}{12}{30} \\
\ganttbar{Obj. 4: Documentation}{30}{36}
\end{ganttchart}
\caption{\label{fig:Gannt-chart}Example of a Gantt chart.}
\end{sidewaysfigure}
%

\section{\label{sec:budget}Budget}

If budget is not key within your project, you can eliminate this
section and leave one entire chapter dedicated to the description of
your project timeline.
