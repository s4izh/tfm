Your abstract text goes here.  Check your departmental regulations, but generally this should be less than 300 words.  See the beginning of Chapter~\ref{ch:background} for more.


The tensor-vector contraction (TVC) is the most memory-bound operation of its
class and a core component of the higher-order power method (HOPM). Our
research team has recently proposed state-of-the-art versions of these two
algorithms for both shared-memory and distributed-memory systems running with
CPUs. This Master's thesis seeks to accelerate such operations on GPUs.

Tensors can be considered as multidimensional arrays that store data in a
certain manner according to multiple (i.e. multilinear) attributes. Exploiting
them is therefore of great importance since it allows us to extract patterns
inherently present in such datasets. A major kernel is the tensor contraction,
which includes the following operations: (i) tensor-tensor contraction (TTC),
(ii) tensor-matrix contraction (TMC), and (iii) tensor-vector contraction
(TVC), being all of them core components in widely used tensor algorithms like
the HOPM.


In this Master's thesis, the student will be tasked to familiarise with the
dTVC library developed by our research group. Secondly, he or she will carry
out the porting of the library to GPUs using the domain-specific language (DSL)
Triton that will largely simplify the entire process. Thirdly, he or she will
evaluate the Triton implementation against the current state-of-the-art
approach: looped TVC algorithm (based on parallel matrix-vector
multiplications) and a naive HOPM implementation. This stage will allow us to
identify possible bottlenecks and areas requiring further improvements.

Fourthly, the student will develop a CUDA implementation of the dTVC library
and benchmark it against the Triton version to achieve even higher speedups.


This Master's thesis will take place at the Departament d'Arquitectura de
Computadors (DAC) at UPC in collaboration with the STAR research group at the
Barcelona Supercomputing Center (BSC). The System Tools and Advanced Runtimes
(STAR) group focuses on research crossing multiple software layers from OS,
runtimes, and low-level APIs, to programming models, tools, and applications.

The group runs traditional HPC, AI, and data analytics applications on
massively parallel HPC and cloud infrastructures.

The student will have the support from researchers within the STAR group, which
will provide them guidance throughout the duration of the thesis. The dTVC
library will be benchmarked using the computational resources of the university
department.

